\documentclass[]{article}
\usepackage{lmodern}
\usepackage{amssymb,amsmath}
\usepackage{ifxetex,ifluatex}
\usepackage{fixltx2e} % provides \textsubscript
\ifnum 0\ifxetex 1\fi\ifluatex 1\fi=0 % if pdftex
  \usepackage[T1]{fontenc}
  \usepackage[utf8]{inputenc}
\else % if luatex or xelatex
  \ifxetex
    \usepackage{mathspec}
  \else
    \usepackage{fontspec}
  \fi
  \defaultfontfeatures{Ligatures=TeX,Scale=MatchLowercase}
\fi
% use upquote if available, for straight quotes in verbatim environments
\IfFileExists{upquote.sty}{\usepackage{upquote}}{}
% use microtype if available
\IfFileExists{microtype.sty}{%
\usepackage{microtype}
\UseMicrotypeSet[protrusion]{basicmath} % disable protrusion for tt fonts
}{}
\usepackage[margin=1in]{geometry}
\usepackage{hyperref}
\hypersetup{unicode=true,
            pdftitle={Computational Modeling - Week 3 - Assignment 2 - Part 1},
            pdfauthor={Riccardo Fusaroli},
            pdfborder={0 0 0},
            breaklinks=true}
\urlstyle{same}  % don't use monospace font for urls
\usepackage{color}
\usepackage{fancyvrb}
\newcommand{\VerbBar}{|}
\newcommand{\VERB}{\Verb[commandchars=\\\{\}]}
\DefineVerbatimEnvironment{Highlighting}{Verbatim}{commandchars=\\\{\}}
% Add ',fontsize=\small' for more characters per line
\usepackage{framed}
\definecolor{shadecolor}{RGB}{248,248,248}
\newenvironment{Shaded}{\begin{snugshade}}{\end{snugshade}}
\newcommand{\KeywordTok}[1]{\textcolor[rgb]{0.13,0.29,0.53}{\textbf{#1}}}
\newcommand{\DataTypeTok}[1]{\textcolor[rgb]{0.13,0.29,0.53}{#1}}
\newcommand{\DecValTok}[1]{\textcolor[rgb]{0.00,0.00,0.81}{#1}}
\newcommand{\BaseNTok}[1]{\textcolor[rgb]{0.00,0.00,0.81}{#1}}
\newcommand{\FloatTok}[1]{\textcolor[rgb]{0.00,0.00,0.81}{#1}}
\newcommand{\ConstantTok}[1]{\textcolor[rgb]{0.00,0.00,0.00}{#1}}
\newcommand{\CharTok}[1]{\textcolor[rgb]{0.31,0.60,0.02}{#1}}
\newcommand{\SpecialCharTok}[1]{\textcolor[rgb]{0.00,0.00,0.00}{#1}}
\newcommand{\StringTok}[1]{\textcolor[rgb]{0.31,0.60,0.02}{#1}}
\newcommand{\VerbatimStringTok}[1]{\textcolor[rgb]{0.31,0.60,0.02}{#1}}
\newcommand{\SpecialStringTok}[1]{\textcolor[rgb]{0.31,0.60,0.02}{#1}}
\newcommand{\ImportTok}[1]{#1}
\newcommand{\CommentTok}[1]{\textcolor[rgb]{0.56,0.35,0.01}{\textit{#1}}}
\newcommand{\DocumentationTok}[1]{\textcolor[rgb]{0.56,0.35,0.01}{\textbf{\textit{#1}}}}
\newcommand{\AnnotationTok}[1]{\textcolor[rgb]{0.56,0.35,0.01}{\textbf{\textit{#1}}}}
\newcommand{\CommentVarTok}[1]{\textcolor[rgb]{0.56,0.35,0.01}{\textbf{\textit{#1}}}}
\newcommand{\OtherTok}[1]{\textcolor[rgb]{0.56,0.35,0.01}{#1}}
\newcommand{\FunctionTok}[1]{\textcolor[rgb]{0.00,0.00,0.00}{#1}}
\newcommand{\VariableTok}[1]{\textcolor[rgb]{0.00,0.00,0.00}{#1}}
\newcommand{\ControlFlowTok}[1]{\textcolor[rgb]{0.13,0.29,0.53}{\textbf{#1}}}
\newcommand{\OperatorTok}[1]{\textcolor[rgb]{0.81,0.36,0.00}{\textbf{#1}}}
\newcommand{\BuiltInTok}[1]{#1}
\newcommand{\ExtensionTok}[1]{#1}
\newcommand{\PreprocessorTok}[1]{\textcolor[rgb]{0.56,0.35,0.01}{\textit{#1}}}
\newcommand{\AttributeTok}[1]{\textcolor[rgb]{0.77,0.63,0.00}{#1}}
\newcommand{\RegionMarkerTok}[1]{#1}
\newcommand{\InformationTok}[1]{\textcolor[rgb]{0.56,0.35,0.01}{\textbf{\textit{#1}}}}
\newcommand{\WarningTok}[1]{\textcolor[rgb]{0.56,0.35,0.01}{\textbf{\textit{#1}}}}
\newcommand{\AlertTok}[1]{\textcolor[rgb]{0.94,0.16,0.16}{#1}}
\newcommand{\ErrorTok}[1]{\textcolor[rgb]{0.64,0.00,0.00}{\textbf{#1}}}
\newcommand{\NormalTok}[1]{#1}
\usepackage{graphicx,grffile}
\makeatletter
\def\maxwidth{\ifdim\Gin@nat@width>\linewidth\linewidth\else\Gin@nat@width\fi}
\def\maxheight{\ifdim\Gin@nat@height>\textheight\textheight\else\Gin@nat@height\fi}
\makeatother
% Scale images if necessary, so that they will not overflow the page
% margins by default, and it is still possible to overwrite the defaults
% using explicit options in \includegraphics[width, height, ...]{}
\setkeys{Gin}{width=\maxwidth,height=\maxheight,keepaspectratio}
\IfFileExists{parskip.sty}{%
\usepackage{parskip}
}{% else
\setlength{\parindent}{0pt}
\setlength{\parskip}{6pt plus 2pt minus 1pt}
}
\setlength{\emergencystretch}{3em}  % prevent overfull lines
\providecommand{\tightlist}{%
  \setlength{\itemsep}{0pt}\setlength{\parskip}{0pt}}
\setcounter{secnumdepth}{0}
% Redefines (sub)paragraphs to behave more like sections
\ifx\paragraph\undefined\else
\let\oldparagraph\paragraph
\renewcommand{\paragraph}[1]{\oldparagraph{#1}\mbox{}}
\fi
\ifx\subparagraph\undefined\else
\let\oldsubparagraph\subparagraph
\renewcommand{\subparagraph}[1]{\oldsubparagraph{#1}\mbox{}}
\fi

%%% Use protect on footnotes to avoid problems with footnotes in titles
\let\rmarkdownfootnote\footnote%
\def\footnote{\protect\rmarkdownfootnote}

%%% Change title format to be more compact
\usepackage{titling}

% Create subtitle command for use in maketitle
\newcommand{\subtitle}[1]{
  \posttitle{
    \begin{center}\large#1\end{center}
    }
}

\setlength{\droptitle}{-2em}

  \title{Computational Modeling - Week 3 - Assignment 2 - Part 1}
    \pretitle{\vspace{\droptitle}\centering\huge}
  \posttitle{\par}
    \author{Riccardo Fusaroli}
    \preauthor{\centering\large\emph}
  \postauthor{\par}
    \date{}
    \predate{}\postdate{}
  

\begin{document}
\maketitle

\subsection{In this assignment we learn how to assess rates from a
binomial distribution, using the case of assessing your teachers'
knowledge of
CogSci}\label{in-this-assignment-we-learn-how-to-assess-rates-from-a-binomial-distribution-using-the-case-of-assessing-your-teachers-knowledge-of-cogsci}

\subsubsection{First part}\label{first-part}

You want to assess your teachers' knowledge of cognitive science.
``These guys are a bunch of drama(turgist) queens, mindless
philosophers, chattering communication people and Russian spies. Do they
really know CogSci?'', you think.

To keep things simple (your teachers should not be faced with too
complicated things): - You created a pool of equally challenging
questions on CogSci - Each question can be answered correctly or not (we
don't allow partially correct answers, to make our life simpler). -
Knowledge of CogSci can be measured on a scale from 0 (negative
knowledge, all answers wrong) through 0.5 (random chance) to 1 (awesome
CogSci superpowers)

This is the data: - Riccardo: 3 correct answers out of 6 questions -
Kristian: 2 correct answers out of 2 questions (then he gets bored) -
Josh: 160 correct answers out of 198 questions (Josh never gets bored) -
Mikkel: 66 correct answers out of 132 questions

Questions:

\section{1. What's Riccardo's estimated knowledge of CogSci? What is the
probability he knows more than chance (0.5) {[}try figuring this out. if
you can't peek into chapters 3.1 and 3.2 and/or the
slides{]}?}\label{whats-riccardos-estimated-knowledge-of-cogsci-what-is-the-probability-he-knows-more-than-chance-0.5-try-figuring-this-out.-if-you-cant-peek-into-chapters-3.1-and-3.2-andor-the-slides}

\begin{Shaded}
\begin{Highlighting}[]
\KeywordTok{sum}\NormalTok{( samples_ric }\OperatorTok{>}\StringTok{ }\FloatTok{0.5}\NormalTok{ ) }\OperatorTok{/}\StringTok{ }\KeywordTok{length}\NormalTok{ (samples_ric) }
\end{Highlighting}
\end{Shaded}

\begin{verbatim}
## [1] 0.5015
\end{verbatim}

The propbability of Riccardo knowing more than chance depends on the
prior. If it is the uniform prior then there is a 50\% chance that he
knows more than 0.5. If we use a prior which assigns 0 to grid-values
below 0.5 and 1 to grid-values above 0.5, then there is a 100\% chance
that he performs above chance.

\begin{itemize}
\tightlist
\item
  First implement a grid approximation (hint check paragraph 2.4.1!)
  with a uniform prior, calculate the posterior and plot the results
\end{itemize}

\begin{Shaded}
\begin{Highlighting}[]
\CommentTok{#plot}
\NormalTok{ric_plot <-}\StringTok{ }\KeywordTok{plot}\NormalTok{( grid , posterior , }\DataTypeTok{type=}\StringTok{"b"}\NormalTok{ ,}
    \DataTypeTok{xlab=}\StringTok{"probability of correct"}\NormalTok{ , }\DataTypeTok{ylab=}\StringTok{"posterior probability"}\NormalTok{ )}
\KeywordTok{mtext}\NormalTok{(}\StringTok{"Riccardo, uniform prior"}\NormalTok{)}
\end{Highlighting}
\end{Shaded}

\includegraphics{Thea_Assignment2_Part1_files/figure-latex/unnamed-chunk-3-1.pdf}

\begin{itemize}
\tightlist
\item
  Then implement a quadratic approximation (hint check paragraph
  2.4.2!).
\end{itemize}

\begin{Shaded}
\begin{Highlighting}[]
\CommentTok{# analytical calculation}
\NormalTok{w <-}\StringTok{ }\DecValTok{3}
\NormalTok{n <-}\StringTok{ }\DecValTok{6}
\KeywordTok{curve}\NormalTok{( }\KeywordTok{dbeta}\NormalTok{( x , w}\OperatorTok{+}\DecValTok{1}\NormalTok{ , n}\OperatorTok{-}\NormalTok{w}\OperatorTok{+}\DecValTok{1}\NormalTok{ ) , }\DataTypeTok{from=}\DecValTok{0}\NormalTok{ , }\DataTypeTok{to=}\DecValTok{1}\NormalTok{ )}
\CommentTok{# quadratic approximation}
\KeywordTok{curve}\NormalTok{( }\KeywordTok{dnorm}\NormalTok{( x , }\FloatTok{0.5}\NormalTok{ , }\FloatTok{0.2}\NormalTok{ ) , }\DataTypeTok{lty=}\DecValTok{2}\NormalTok{ , }\DataTypeTok{add=}\OtherTok{TRUE}\NormalTok{ )}
\KeywordTok{mtext}\NormalTok{(}\StringTok{"Quadratic approximation: prior with assumption of knowledge > 0.5"}\NormalTok{)}
\end{Highlighting}
\end{Shaded}

\includegraphics{Thea_Assignment2_Part1_files/figure-latex/unnamed-chunk-5-1.pdf}

N.B. for the rest of the exercise just keep using the grid approximation
(we'll move to quadratic approximations in two classes)

\section{2. Estimate all the teachers' knowledge of CogSci. Who's best?
Use grid approximation. Comment on the posteriors of Riccardo and
Mikkel.}\label{estimate-all-the-teachers-knowledge-of-cogsci.-whos-best-use-grid-approximation.-comment-on-the-posteriors-of-riccardo-and-mikkel.}

2a. Produce plots of the prior, and posterior for each teacher.

From now on we will use the following prior that assumes the teachers
will perform better than chance (which is basically the same as having
zero CogSci knowledge)

\begin{Shaded}
\begin{Highlighting}[]
\CommentTok{#plot of prior}

\NormalTok{Data=}\KeywordTok{data.frame}\NormalTok{(}\DataTypeTok{grid=}\NormalTok{grid,}\DataTypeTok{posterior=}\NormalTok{posterior,}\DataTypeTok{prior=}\NormalTok{alt_prior,}\DataTypeTok{likelihood=}\NormalTok{likelihood)}
\KeywordTok{ggplot}\NormalTok{(Data,}\KeywordTok{aes}\NormalTok{(grid,posterior)) }\OperatorTok{+}\StringTok{ }\KeywordTok{geom_line}\NormalTok{(}\KeywordTok{aes}\NormalTok{(grid,alt_prior),}\DataTypeTok{color=}\StringTok{'red'}\NormalTok{)}\OperatorTok{+}\StringTok{ }\KeywordTok{xlab}\NormalTok{(}\StringTok{"Knowledge of CogSci"}\NormalTok{)}\OperatorTok{+}\StringTok{ }\KeywordTok{ylab}\NormalTok{(}\StringTok{"posterior probability"}\NormalTok{) }\OperatorTok{+}\StringTok{ }\KeywordTok{labs}\NormalTok{(}\DataTypeTok{title =} \StringTok{"Prior with assumption of knowledge > 0.5"}\NormalTok{)}
\end{Highlighting}
\end{Shaded}

\includegraphics{Thea_Assignment2_Part1_files/figure-latex/unnamed-chunk-6-1.pdf}

\begin{Shaded}
\begin{Highlighting}[]
\CommentTok{# tylle}

\CommentTok{# compute likelihood at each value in grid}
\NormalTok{likelihood <-}\StringTok{ }\KeywordTok{dbinom}\NormalTok{( }\DecValTok{2}\NormalTok{ , }\DataTypeTok{size=}\DecValTok{2}\NormalTok{ , }\DataTypeTok{prob=}\NormalTok{grid )}
\CommentTok{# compute product of likelihood and prior}
\NormalTok{unstd.posterior <-}\StringTok{ }\NormalTok{likelihood }\OperatorTok{*}\StringTok{ }\NormalTok{alt_prior}
\CommentTok{# standardize the posterior, so it sums to 1}
\NormalTok{posterior <-}\StringTok{ }\NormalTok{unstd.posterior }\OperatorTok{/}\StringTok{ }\KeywordTok{sum}\NormalTok{(unstd.posterior)}

\CommentTok{#plot}
\NormalTok{tylle_plot <-}\StringTok{ }\KeywordTok{plot}\NormalTok{( grid , posterior , }\DataTypeTok{type=}\StringTok{"b"}\NormalTok{ ,}
    \DataTypeTok{xlab=}\StringTok{"probability of correct"}\NormalTok{ , }\DataTypeTok{ylab=}\StringTok{"posterior probability"}\NormalTok{ )}
\KeywordTok{mtext}\NormalTok{(}\StringTok{"Tylle, prior with assumption of knowledge > 0.5"}\NormalTok{)}
\end{Highlighting}
\end{Shaded}

\includegraphics{Thea_Assignment2_Part1_files/figure-latex/unnamed-chunk-7-1.pdf}

\begin{Shaded}
\begin{Highlighting}[]
\CommentTok{#Josh}
\CommentTok{# compute likelihood at each value in grid}
\NormalTok{likelihood <-}\StringTok{ }\KeywordTok{dbinom}\NormalTok{( }\DecValTok{160}\NormalTok{ , }\DataTypeTok{size=}\DecValTok{198}\NormalTok{ , }\DataTypeTok{prob=}\NormalTok{grid )}
\CommentTok{# compute product of likelihood and alt_prior}
\NormalTok{unstd.posterior <-}\StringTok{ }\NormalTok{likelihood }\OperatorTok{*}\StringTok{ }\NormalTok{alt_prior}
\CommentTok{# standardize the posterior, so it sums to 1}
\NormalTok{posterior <-}\StringTok{ }\NormalTok{unstd.posterior }\OperatorTok{/}\StringTok{ }\KeywordTok{sum}\NormalTok{(unstd.posterior)}

\CommentTok{#plot}
\NormalTok{josh_plot <-}\StringTok{ }\KeywordTok{plot}\NormalTok{( grid , posterior , }\DataTypeTok{type=}\StringTok{"b"}\NormalTok{ ,}
    \DataTypeTok{xlab=}\StringTok{"probability of correct"}\NormalTok{ , }\DataTypeTok{ylab=}\StringTok{"posterior probability"}\NormalTok{ )}
\KeywordTok{mtext}\NormalTok{(}\StringTok{"Josh, prior with assumption of knowledge > 0.5"}\NormalTok{)}
\end{Highlighting}
\end{Shaded}

\includegraphics{Thea_Assignment2_Part1_files/figure-latex/unnamed-chunk-7-2.pdf}

\begin{Shaded}
\begin{Highlighting}[]
\CommentTok{#Mikkel}
\CommentTok{# compute likelihood at each value in grid}
\NormalTok{likelihood <-}\StringTok{ }\KeywordTok{dbinom}\NormalTok{( }\DecValTok{66}\NormalTok{ , }\DataTypeTok{size=}\DecValTok{132}\NormalTok{ , }\DataTypeTok{prob=}\NormalTok{grid )}
\CommentTok{# compute product of likelihood and alt_prior}
\NormalTok{unstd.posterior <-}\StringTok{ }\NormalTok{likelihood }\OperatorTok{*}\StringTok{ }\NormalTok{alt_prior}
\CommentTok{# standardize the posterior, so it sums to 1}
\NormalTok{posterior <-}\StringTok{ }\NormalTok{unstd.posterior }\OperatorTok{/}\StringTok{ }\KeywordTok{sum}\NormalTok{(unstd.posterior)}

\CommentTok{#plot}
\NormalTok{mikkel_plot <-}\StringTok{ }\KeywordTok{plot}\NormalTok{( grid , posterior , }\DataTypeTok{type=}\StringTok{"b"}\NormalTok{ ,}
    \DataTypeTok{xlab=}\StringTok{"probability of correct"}\NormalTok{ , }\DataTypeTok{ylab=}\StringTok{"posterior probability"}\NormalTok{ )}
\KeywordTok{mtext}\NormalTok{(}\StringTok{"Mikkel, prior with assumption of knowledge > 0.5"}\NormalTok{)}
\end{Highlighting}
\end{Shaded}

\includegraphics{Thea_Assignment2_Part1_files/figure-latex/unnamed-chunk-7-3.pdf}

\begin{Shaded}
\begin{Highlighting}[]
\CommentTok{#Riccardo}
\CommentTok{# compute likelihood at each value in grid}
\NormalTok{likelihood <-}\StringTok{ }\KeywordTok{dbinom}\NormalTok{( }\DecValTok{3}\NormalTok{ , }\DataTypeTok{size=}\DecValTok{6}\NormalTok{ , }\DataTypeTok{prob=}\NormalTok{grid )}
\CommentTok{# compute product of likelihood and alt_prior}
\NormalTok{unstd.posterior <-}\StringTok{ }\NormalTok{likelihood }\OperatorTok{*}\StringTok{ }\NormalTok{alt_prior}
\CommentTok{# standardize the posterior, so it sums to 1}
\NormalTok{posterior <-}\StringTok{ }\NormalTok{unstd.posterior }\OperatorTok{/}\StringTok{ }\KeywordTok{sum}\NormalTok{(unstd.posterior)}

\CommentTok{#plot}
\NormalTok{riccardo_plot <-}\StringTok{ }\KeywordTok{plot}\NormalTok{( grid , posterior , }\DataTypeTok{type=}\StringTok{"b"}\NormalTok{ ,}
    \DataTypeTok{xlab=}\StringTok{"probability of correct"}\NormalTok{ , }\DataTypeTok{ylab=}\StringTok{"posterior probability"}\NormalTok{ )}
\KeywordTok{mtext}\NormalTok{(}\StringTok{"Riccardo, prior with assumption of knowledge > 0.5"}\NormalTok{)}
\end{Highlighting}
\end{Shaded}

\includegraphics{Thea_Assignment2_Part1_files/figure-latex/unnamed-chunk-7-4.pdf}

In percentages Riccardo and Mikkel answer roughly equally bad. However,
since we have many more datapoints from Mikkel, as the posterior
distribution shows, we can be much more certain in assessing Mikkel's
CogSci knowledge. This is shown in the area under the curve being more
narrow compared to Riccardo's.

\section{Who's best?}\label{whos-best}

Assessing the posterior distribution, we are quite certrain that Josh's
knowledge lays around 0.8, which is somewhat impressive. However,
looking at Tylle's curve we are very uncertain - it could be 0.5, but
also 1 (awesome). This is due to the fact that we have sucj few
datapoints from Tylle, resulting in high uncertainty.

\section{3. Change the prior. Given your teachers have all CogSci jobs,
you should start with a higher appreciation of their knowledge: the
prior is a normal distribution with a mean of 0.8 and a standard
deviation of 0.2. Do the results change (and if so
how)?}\label{change-the-prior.-given-your-teachers-have-all-cogsci-jobs-you-should-start-with-a-higher-appreciation-of-their-knowledge-the-prior-is-a-normal-distribution-with-a-mean-of-0.8-and-a-standard-deviation-of-0.2.-do-the-results-change-and-if-so-how}

3a. Produce plots of the prior and posterior for each teacher.

\begin{Shaded}
\begin{Highlighting}[]
\NormalTok{new_prior <-}\StringTok{ }\KeywordTok{dnorm}\NormalTok{(grid,}\FloatTok{0.8}\NormalTok{,}\FloatTok{0.2}\NormalTok{)}
\CommentTok{# plot(new_prior)}
\CommentTok{#plot of new prior}

\NormalTok{new_Data=}\KeywordTok{data.frame}\NormalTok{(}\DataTypeTok{grid=}\NormalTok{grid,}\DataTypeTok{posterior=}\NormalTok{posterior,}\DataTypeTok{prior=}\NormalTok{new_prior,}\DataTypeTok{likelihood=}\NormalTok{likelihood)}
\KeywordTok{ggplot}\NormalTok{(new_Data,}\KeywordTok{aes}\NormalTok{(grid,posterior)) }\OperatorTok{+}\StringTok{ }\KeywordTok{geom_line}\NormalTok{(}\KeywordTok{aes}\NormalTok{(grid,prior),}\DataTypeTok{color=}\StringTok{'red'}\NormalTok{)}\OperatorTok{+}\StringTok{ }\KeywordTok{xlab}\NormalTok{(}\StringTok{"Knowledge of CogSci"}\NormalTok{)}\OperatorTok{+}\StringTok{ }\KeywordTok{ylab}\NormalTok{(}\StringTok{"posterior probability"}\NormalTok{) }\OperatorTok{+}\StringTok{ }\KeywordTok{labs}\NormalTok{(}\DataTypeTok{title =} \StringTok{"Plot of new normal prior (0.8,0.2)"}\NormalTok{)}
\end{Highlighting}
\end{Shaded}

\includegraphics{Thea_Assignment2_Part1_files/figure-latex/unnamed-chunk-8-1.pdf}

\begin{Shaded}
\begin{Highlighting}[]
\CommentTok{#posterior for each teacher}

\CommentTok{# tylle}

\CommentTok{# compute likelihood at each value in grid}
\NormalTok{likelihood <-}\StringTok{ }\KeywordTok{dbinom}\NormalTok{( }\DecValTok{2}\NormalTok{ , }\DataTypeTok{size=}\DecValTok{2}\NormalTok{ , }\DataTypeTok{prob=}\NormalTok{grid )}
\CommentTok{# compute product of likelihood and prior}
\NormalTok{unstd.posterior <-}\StringTok{ }\NormalTok{likelihood }\OperatorTok{*}\StringTok{ }\NormalTok{new_prior}
\CommentTok{# standardize the posterior, so it sums to 1}
\NormalTok{posterior <-}\StringTok{ }\NormalTok{unstd.posterior }\OperatorTok{/}\StringTok{ }\KeywordTok{sum}\NormalTok{(unstd.posterior)}

\CommentTok{#plot}
\NormalTok{tylle_plot <-}\StringTok{ }\KeywordTok{plot}\NormalTok{( grid , posterior , }\DataTypeTok{type=}\StringTok{"b"}\NormalTok{ ,}
    \DataTypeTok{xlab=}\StringTok{"probability of correct"}\NormalTok{ , }\DataTypeTok{ylab=}\StringTok{"posterior probability"}\NormalTok{ )}
\KeywordTok{mtext}\NormalTok{(}\StringTok{"Tylle, new normal prior (0.8,0.2)"}\NormalTok{)}
\end{Highlighting}
\end{Shaded}

\includegraphics{Thea_Assignment2_Part1_files/figure-latex/unnamed-chunk-9-1.pdf}

\begin{Shaded}
\begin{Highlighting}[]
\CommentTok{#Josh}
\CommentTok{# compute likelihood at each value in grid}
\NormalTok{likelihood <-}\StringTok{ }\KeywordTok{dbinom}\NormalTok{( }\DecValTok{160}\NormalTok{ , }\DataTypeTok{size=}\DecValTok{198}\NormalTok{ , }\DataTypeTok{prob=}\NormalTok{grid )}
\CommentTok{# compute product of likelihood and new_prior}
\NormalTok{unstd.posterior <-}\StringTok{ }\NormalTok{likelihood }\OperatorTok{*}\StringTok{ }\NormalTok{new_prior}
\CommentTok{# standardize the posterior, so it sums to 1}
\NormalTok{posterior <-}\StringTok{ }\NormalTok{unstd.posterior }\OperatorTok{/}\StringTok{ }\KeywordTok{sum}\NormalTok{(unstd.posterior)}

\CommentTok{#plot}
\NormalTok{josh_plot <-}\StringTok{ }\KeywordTok{plot}\NormalTok{( grid , posterior , }\DataTypeTok{type=}\StringTok{"b"}\NormalTok{ ,}
    \DataTypeTok{xlab=}\StringTok{"probability of correct"}\NormalTok{ , }\DataTypeTok{ylab=}\StringTok{"posterior probability"}\NormalTok{ )}
\KeywordTok{mtext}\NormalTok{(}\StringTok{"Josh, new normal prior (0.8,0.2)"}\NormalTok{)}
\end{Highlighting}
\end{Shaded}

\includegraphics{Thea_Assignment2_Part1_files/figure-latex/unnamed-chunk-9-2.pdf}

\begin{Shaded}
\begin{Highlighting}[]
\CommentTok{#Mikkel}
\CommentTok{# compute likelihood at each value in grid}
\NormalTok{likelihood <-}\StringTok{ }\KeywordTok{dbinom}\NormalTok{( }\DecValTok{66}\NormalTok{ , }\DataTypeTok{size=}\DecValTok{132}\NormalTok{ , }\DataTypeTok{prob=}\NormalTok{grid )}
\CommentTok{# compute product of likelihood and new_prior}
\NormalTok{unstd.posterior <-}\StringTok{ }\NormalTok{likelihood }\OperatorTok{*}\StringTok{ }\NormalTok{new_prior}
\CommentTok{# standardize the posterior, so it sums to 1}
\NormalTok{posterior <-}\StringTok{ }\NormalTok{unstd.posterior }\OperatorTok{/}\StringTok{ }\KeywordTok{sum}\NormalTok{(unstd.posterior)}

\CommentTok{#plot}
\NormalTok{mikkel_plot <-}\StringTok{ }\KeywordTok{plot}\NormalTok{( grid , posterior , }\DataTypeTok{type=}\StringTok{"b"}\NormalTok{ ,}
    \DataTypeTok{xlab=}\StringTok{"probability of correct"}\NormalTok{ , }\DataTypeTok{ylab=}\StringTok{"posterior probability"}\NormalTok{ )}
\KeywordTok{mtext}\NormalTok{(}\StringTok{"Mikkel, new normal prior (0.8,0.2)"}\NormalTok{)}
\end{Highlighting}
\end{Shaded}

\includegraphics{Thea_Assignment2_Part1_files/figure-latex/unnamed-chunk-9-3.pdf}

\begin{Shaded}
\begin{Highlighting}[]
\CommentTok{#riccardo}
\CommentTok{# compute likelihood at each value in grid}
\NormalTok{likelihood <-}\StringTok{ }\KeywordTok{dbinom}\NormalTok{( }\DecValTok{3}\NormalTok{ , }\DataTypeTok{size=}\DecValTok{6}\NormalTok{ , }\DataTypeTok{prob=}\NormalTok{grid )}
\CommentTok{# compute product of likelihood and new_prior}
\NormalTok{unstd.posterior <-}\StringTok{ }\NormalTok{likelihood }\OperatorTok{*}\StringTok{ }\NormalTok{new_prior}
\CommentTok{# standardize the posterior, so it sums to 1}
\NormalTok{posterior <-}\StringTok{ }\NormalTok{unstd.posterior }\OperatorTok{/}\StringTok{ }\KeywordTok{sum}\NormalTok{(unstd.posterior)}

\CommentTok{#plot}
\NormalTok{ric_plot <-}\StringTok{ }\KeywordTok{plot}\NormalTok{( grid , posterior , }\DataTypeTok{type=}\StringTok{"b"}\NormalTok{ ,}
    \DataTypeTok{xlab=}\StringTok{"probability of correct"}\NormalTok{ , }\DataTypeTok{ylab=}\StringTok{"posterior probability"}\NormalTok{ )}
\KeywordTok{mtext}\NormalTok{(}\StringTok{"Riccardo, new normal prior (0.8,0.2)"}\NormalTok{)}
\end{Highlighting}
\end{Shaded}

\includegraphics{Thea_Assignment2_Part1_files/figure-latex/unnamed-chunk-9-4.pdf}

Change in results: Yes, they change. At least for Tylle and Riccardo. We
can see that the prior affects the posterior more, when there are few
datapoints. Josh and Mikkel remain more or less the samme, since there
are many datapoints, and thereby are not so manipulative by the prior.

\section{4. You go back to your teachers and collect more data (multiply
the previous numbers by 100). Calculate their knowledge with both a
uniform prior and a normal prior with a mean of 0.8 and a standard
deviation of 0.2. Do you still see a difference between the results?
Why?}\label{you-go-back-to-your-teachers-and-collect-more-data-multiply-the-previous-numbers-by-100.-calculate-their-knowledge-with-both-a-uniform-prior-and-a-normal-prior-with-a-mean-of-0.8-and-a-standard-deviation-of-0.2.-do-you-still-see-a-difference-between-the-results-why}

\begin{Shaded}
\begin{Highlighting}[]
\CommentTok{#multiply all data with 100}
\NormalTok{new_d <-}\StringTok{ }\NormalTok{d}\OperatorTok{*}\DecValTok{100}
\end{Highlighting}
\end{Shaded}

\begin{verbatim}
## Warning in Ops.factor(left, right): '*' not meaningful for factors
\end{verbatim}

\begin{Shaded}
\begin{Highlighting}[]
\NormalTok{new_d}\OperatorTok{$}\NormalTok{Teacher <-}\StringTok{ }\NormalTok{d}\OperatorTok{$}\NormalTok{Teacher}

\NormalTok{new_d}
\end{Highlighting}
\end{Shaded}

\begin{verbatim}
##   Correct Questions Teacher
## 1     300       600      RF
## 2     200       200      KT
## 3   16000     19800      JS
## 4    6600     13200      MW
\end{verbatim}

Uniform prior

\begin{Shaded}
\begin{Highlighting}[]
\CommentTok{#posterior for each teacher}

\CommentTok{# tylle}

\CommentTok{# compute likelihood at each value in grid}
\NormalTok{likelihood <-}\StringTok{ }\KeywordTok{dbinom}\NormalTok{( new_d[}\DecValTok{2}\NormalTok{,}\DecValTok{1}\NormalTok{] , }\DataTypeTok{size=}\NormalTok{ new_d[}\DecValTok{2}\NormalTok{,}\DecValTok{1}\NormalTok{], }\DataTypeTok{prob=}\NormalTok{grid )}
\CommentTok{# compute product of likelihood and prior}
\NormalTok{unstd.posterior <-}\StringTok{ }\NormalTok{likelihood }\OperatorTok{*}\StringTok{ }\NormalTok{uni_prior}
\CommentTok{# standardize the posterior, so it sums to 1}
\NormalTok{posterior <-}\StringTok{ }\NormalTok{unstd.posterior }\OperatorTok{/}\StringTok{ }\KeywordTok{sum}\NormalTok{(unstd.posterior)}

\CommentTok{#plot}
\NormalTok{tylle_plot <-}\StringTok{ }\KeywordTok{plot}\NormalTok{( grid , posterior , }\DataTypeTok{type=}\StringTok{"b"}\NormalTok{ ,}
    \DataTypeTok{xlab=}\StringTok{"probability of correct"}\NormalTok{ , }\DataTypeTok{ylab=}\StringTok{"posterior probability"}\NormalTok{ )}
\KeywordTok{mtext}\NormalTok{(}\StringTok{"Tylle with more data, uniform prior"}\NormalTok{)}
\end{Highlighting}
\end{Shaded}

\includegraphics{Thea_Assignment2_Part1_files/figure-latex/unnamed-chunk-12-1.pdf}

\begin{Shaded}
\begin{Highlighting}[]
\CommentTok{#Josh}
\CommentTok{# compute likelihood at each value in grid}
\NormalTok{likelihood <-}\StringTok{ }\KeywordTok{dbinom}\NormalTok{( new_d[}\DecValTok{3}\NormalTok{,}\DecValTok{1}\NormalTok{] , }\DataTypeTok{size=}\NormalTok{new_d[}\DecValTok{3}\NormalTok{,}\DecValTok{2}\NormalTok{] , }\DataTypeTok{prob=}\NormalTok{grid )}
\CommentTok{# compute product of likelihood and new_prior}
\NormalTok{unstd.posterior <-}\StringTok{ }\NormalTok{likelihood }\OperatorTok{*}\StringTok{ }\NormalTok{uni_prior}
\CommentTok{# standardize the posterior, so it sums to 1}
\NormalTok{posterior <-}\StringTok{ }\NormalTok{unstd.posterior }\OperatorTok{/}\StringTok{ }\KeywordTok{sum}\NormalTok{(unstd.posterior)}

\CommentTok{#plot}
\NormalTok{josh_plot <-}\StringTok{ }\KeywordTok{plot}\NormalTok{( grid , posterior , }\DataTypeTok{type=}\StringTok{"b"}\NormalTok{ ,}
    \DataTypeTok{xlab=}\StringTok{"probability of correct"}\NormalTok{ , }\DataTypeTok{ylab=}\StringTok{"posterior probability"}\NormalTok{ )}
\KeywordTok{mtext}\NormalTok{(}\StringTok{"Josh with more data, uniform prior"}\NormalTok{)}
\end{Highlighting}
\end{Shaded}

\includegraphics{Thea_Assignment2_Part1_files/figure-latex/unnamed-chunk-12-2.pdf}

\begin{Shaded}
\begin{Highlighting}[]
\CommentTok{#Mikkel}
\CommentTok{# compute likelihood at each value in grid}
\NormalTok{likelihood <-}\StringTok{ }\KeywordTok{dbinom}\NormalTok{( new_d[}\DecValTok{4}\NormalTok{,}\DecValTok{1}\NormalTok{] , }\DataTypeTok{size=}\NormalTok{new_d[}\DecValTok{4}\NormalTok{,}\DecValTok{2}\NormalTok{] , }\DataTypeTok{prob=}\NormalTok{grid )}
\CommentTok{# compute product of likelihood and new_prior}
\NormalTok{unstd.posterior <-}\StringTok{ }\NormalTok{likelihood }\OperatorTok{*}\StringTok{ }\NormalTok{uni_prior}
\CommentTok{# standardize the posterior, so it sums to 1}
\NormalTok{posterior <-}\StringTok{ }\NormalTok{unstd.posterior }\OperatorTok{/}\StringTok{ }\KeywordTok{sum}\NormalTok{(unstd.posterior)}

\CommentTok{#plot}
\NormalTok{mikkel_plot <-}\StringTok{ }\KeywordTok{plot}\NormalTok{( grid , posterior , }\DataTypeTok{type=}\StringTok{"b"}\NormalTok{ ,}
    \DataTypeTok{xlab=}\StringTok{"probability of correct"}\NormalTok{ , }\DataTypeTok{ylab=}\StringTok{"posterior probability"}\NormalTok{ )}
\KeywordTok{mtext}\NormalTok{(}\StringTok{"Mikkel with more data, uniform prior"}\NormalTok{)}
\end{Highlighting}
\end{Shaded}

\includegraphics{Thea_Assignment2_Part1_files/figure-latex/unnamed-chunk-12-3.pdf}

\begin{Shaded}
\begin{Highlighting}[]
\CommentTok{#riccardo}
\CommentTok{# compute likelihood at each value in grid}
\NormalTok{likelihood <-}\StringTok{ }\KeywordTok{dbinom}\NormalTok{( new_d[}\DecValTok{1}\NormalTok{,}\DecValTok{1}\NormalTok{] , }\DataTypeTok{size=}\NormalTok{new_d[}\DecValTok{1}\NormalTok{,}\DecValTok{2}\NormalTok{], }\DataTypeTok{prob=}\NormalTok{grid )}
\CommentTok{# compute product of likelihood and new_prior}
\NormalTok{unstd.posterior <-}\StringTok{ }\NormalTok{likelihood }\OperatorTok{*}\StringTok{ }\NormalTok{uni_prior}
\CommentTok{# standardize the posterior, so it sums to 1}
\NormalTok{posterior <-}\StringTok{ }\NormalTok{unstd.posterior }\OperatorTok{/}\StringTok{ }\KeywordTok{sum}\NormalTok{(unstd.posterior)}

\CommentTok{#plot}
\NormalTok{ric_plot <-}\StringTok{ }\KeywordTok{plot}\NormalTok{( grid , posterior , }\DataTypeTok{type=}\StringTok{"b"}\NormalTok{ ,}
    \DataTypeTok{xlab=}\StringTok{"probability of correct"}\NormalTok{ , }\DataTypeTok{ylab=}\StringTok{"posterior probability"}\NormalTok{ )}
\KeywordTok{mtext}\NormalTok{(}\StringTok{"Riccardo with more data, uniform prior"}\NormalTok{)}
\end{Highlighting}
\end{Shaded}

\includegraphics{Thea_Assignment2_Part1_files/figure-latex/unnamed-chunk-12-4.pdf}

Normal prior:

\begin{Shaded}
\begin{Highlighting}[]
\CommentTok{#posterior for each teacher}

\CommentTok{# tylle}

\CommentTok{# compute likelihood at each value in grid}
\NormalTok{likelihood <-}\StringTok{ }\KeywordTok{dbinom}\NormalTok{( new_d[}\DecValTok{2}\NormalTok{,}\DecValTok{1}\NormalTok{] , }\DataTypeTok{size=}\NormalTok{ new_d[}\DecValTok{2}\NormalTok{,}\DecValTok{1}\NormalTok{], }\DataTypeTok{prob=}\NormalTok{grid )}
\CommentTok{# compute product of likelihood and prior}
\NormalTok{unstd.posterior <-}\StringTok{ }\NormalTok{likelihood }\OperatorTok{*}\StringTok{ }\NormalTok{norm_prior}
\CommentTok{# standardize the posterior, so it sums to 1}
\NormalTok{posterior <-}\StringTok{ }\NormalTok{unstd.posterior }\OperatorTok{/}\StringTok{ }\KeywordTok{sum}\NormalTok{(unstd.posterior)}

\CommentTok{#plot}
\NormalTok{tylle_plot <-}\StringTok{ }\KeywordTok{plot}\NormalTok{( grid , posterior , }\DataTypeTok{type=}\StringTok{"b"}\NormalTok{ ,}
    \DataTypeTok{xlab=}\StringTok{"probability of correct"}\NormalTok{ , }\DataTypeTok{ylab=}\StringTok{"posterior probability"}\NormalTok{ )}
\KeywordTok{mtext}\NormalTok{(}\StringTok{"Tylle with more data, new normal prior (0.8,0.2)"}\NormalTok{)}
\end{Highlighting}
\end{Shaded}

\includegraphics{Thea_Assignment2_Part1_files/figure-latex/unnamed-chunk-13-1.pdf}

\begin{Shaded}
\begin{Highlighting}[]
\CommentTok{#Josh}
\CommentTok{# compute likelihood at each value in grid}
\NormalTok{likelihood <-}\StringTok{ }\KeywordTok{dbinom}\NormalTok{( new_d[}\DecValTok{3}\NormalTok{,}\DecValTok{1}\NormalTok{] , }\DataTypeTok{size=}\NormalTok{new_d[}\DecValTok{3}\NormalTok{,}\DecValTok{2}\NormalTok{] , }\DataTypeTok{prob=}\NormalTok{grid )}
\CommentTok{# compute product of likelihood and new_prior}
\NormalTok{unstd.posterior <-}\StringTok{ }\NormalTok{likelihood }\OperatorTok{*}\StringTok{ }\NormalTok{norm_prior}
\CommentTok{# standardize the posterior, so it sums to 1}
\NormalTok{posterior <-}\StringTok{ }\NormalTok{unstd.posterior }\OperatorTok{/}\StringTok{ }\KeywordTok{sum}\NormalTok{(unstd.posterior)}

\CommentTok{#plot}
\NormalTok{josh_plot <-}\StringTok{ }\KeywordTok{plot}\NormalTok{( grid , posterior , }\DataTypeTok{type=}\StringTok{"b"}\NormalTok{ ,}
    \DataTypeTok{xlab=}\StringTok{"probability of correct"}\NormalTok{ , }\DataTypeTok{ylab=}\StringTok{"posterior probability"}\NormalTok{ )}
\KeywordTok{mtext}\NormalTok{(}\StringTok{"Josh with more data, new normal prior (0.8,0.2)"}\NormalTok{)}
\end{Highlighting}
\end{Shaded}

\includegraphics{Thea_Assignment2_Part1_files/figure-latex/unnamed-chunk-13-2.pdf}

\begin{Shaded}
\begin{Highlighting}[]
\CommentTok{#Mikkel}
\CommentTok{# compute likelihood at each value in grid}
\NormalTok{likelihood <-}\StringTok{ }\KeywordTok{dbinom}\NormalTok{( new_d[}\DecValTok{4}\NormalTok{,}\DecValTok{1}\NormalTok{] , }\DataTypeTok{size=}\NormalTok{new_d[}\DecValTok{4}\NormalTok{,}\DecValTok{2}\NormalTok{] , }\DataTypeTok{prob=}\NormalTok{grid )}
\CommentTok{# compute product of likelihood and new_prior}
\NormalTok{unstd.posterior <-}\StringTok{ }\NormalTok{likelihood }\OperatorTok{*}\StringTok{ }\NormalTok{norm_prior}
\CommentTok{# standardize the posterior, so it sums to 1}
\NormalTok{posterior <-}\StringTok{ }\NormalTok{unstd.posterior }\OperatorTok{/}\StringTok{ }\KeywordTok{sum}\NormalTok{(unstd.posterior)}

\CommentTok{#plot}
\NormalTok{mikkel_plot <-}\StringTok{ }\KeywordTok{plot}\NormalTok{( grid , posterior , }\DataTypeTok{type=}\StringTok{"b"}\NormalTok{ ,}
    \DataTypeTok{xlab=}\StringTok{"probability of correct"}\NormalTok{ , }\DataTypeTok{ylab=}\StringTok{"posterior probability"}\NormalTok{ )}
\KeywordTok{mtext}\NormalTok{(}\StringTok{"Mikkel with more data, new normal prior (0.8,0.2)"}\NormalTok{)}
\end{Highlighting}
\end{Shaded}

\includegraphics{Thea_Assignment2_Part1_files/figure-latex/unnamed-chunk-13-3.pdf}

\begin{Shaded}
\begin{Highlighting}[]
\CommentTok{#riccardo}
\CommentTok{# compute likelihood at each value in grid}
\NormalTok{likelihood <-}\StringTok{ }\KeywordTok{dbinom}\NormalTok{( new_d[}\DecValTok{1}\NormalTok{,}\DecValTok{1}\NormalTok{] , }\DataTypeTok{size=}\NormalTok{new_d[}\DecValTok{1}\NormalTok{,}\DecValTok{2}\NormalTok{], }\DataTypeTok{prob=}\NormalTok{grid )}
\CommentTok{# compute product of likelihood and new_prior}
\NormalTok{unstd.posterior <-}\StringTok{ }\NormalTok{likelihood }\OperatorTok{*}\StringTok{ }\NormalTok{norm_prior}
\CommentTok{# standardize the posterior, so it sums to 1}
\NormalTok{posterior <-}\StringTok{ }\NormalTok{unstd.posterior }\OperatorTok{/}\StringTok{ }\KeywordTok{sum}\NormalTok{(unstd.posterior)}

\CommentTok{#plot}
\NormalTok{ric_plot <-}\StringTok{ }\KeywordTok{plot}\NormalTok{( grid , posterior , }\DataTypeTok{type=}\StringTok{"b"}\NormalTok{ ,}
    \DataTypeTok{xlab=}\StringTok{"probability of correct"}\NormalTok{ , }\DataTypeTok{ylab=}\StringTok{"posterior probability"}\NormalTok{ )}
\KeywordTok{mtext}\NormalTok{(}\StringTok{"Riccardo with more data, new normal prior (0.8,0.2)"}\NormalTok{)}
\end{Highlighting}
\end{Shaded}

\includegraphics{Thea_Assignment2_Part1_files/figure-latex/unnamed-chunk-13-4.pdf}

The impact of the prior becomes very small when we have so much data.

\section{5. Imagine you're a skeptic and think your teachers do not know
anything about CogSci, given the content of their classes. How would you
operationalize that
belief?}\label{imagine-youre-a-skeptic-and-think-your-teachers-do-not-know-anything-about-cogsci-given-the-content-of-their-classes.-how-would-you-operationalize-that-belief}

Make a prior of a normal distribution that peaks at 0.5 and with a quite
small SE, since we are pretty sure that they are useless.

\begin{Shaded}
\begin{Highlighting}[]
\NormalTok{stupid_prior <-}\StringTok{ }\KeywordTok{dnorm}\NormalTok{(grid,}\FloatTok{0.5}\NormalTok{,}\FloatTok{0.1}\NormalTok{)}

\NormalTok{new_Data=}\KeywordTok{data.frame}\NormalTok{(}\DataTypeTok{grid=}\NormalTok{grid,}\DataTypeTok{posterior=}\NormalTok{posterior,}\DataTypeTok{prior=}\NormalTok{stupid_prior,}\DataTypeTok{likelihood=}\NormalTok{likelihood)}
\KeywordTok{ggplot}\NormalTok{(new_Data,}\KeywordTok{aes}\NormalTok{(grid,posterior)) }\OperatorTok{+}\StringTok{ }\KeywordTok{geom_line}\NormalTok{(}\KeywordTok{aes}\NormalTok{(grid,stupid_prior),}\DataTypeTok{color=}\StringTok{'red'}\NormalTok{)}\OperatorTok{+}\StringTok{ }\KeywordTok{xlab}\NormalTok{(}\StringTok{"Knowledge of CogSci"}\NormalTok{)}\OperatorTok{+}\StringTok{ }\KeywordTok{ylab}\NormalTok{(}\StringTok{"posterior probability"}\NormalTok{) }\OperatorTok{+}\StringTok{ }\KeywordTok{labs}\NormalTok{(}\DataTypeTok{title =} \StringTok{"Plot of stupid prior"}\NormalTok{)}
\end{Highlighting}
\end{Shaded}

\includegraphics{Thea_Assignment2_Part1_files/figure-latex/unnamed-chunk-14-1.pdf}

\begin{Shaded}
\begin{Highlighting}[]
\CommentTok{# tylle}

\CommentTok{# compute likelihood at each value in grid}
\NormalTok{likelihood <-}\StringTok{ }\KeywordTok{dbinom}\NormalTok{( }\DecValTok{2}\NormalTok{ , }\DataTypeTok{size=}\DecValTok{2}\NormalTok{ , }\DataTypeTok{prob=}\NormalTok{grid )}
\CommentTok{# compute product of likelihood and prior}
\NormalTok{unstd.posterior <-}\StringTok{ }\NormalTok{likelihood }\OperatorTok{*}\StringTok{ }\NormalTok{stupid_prior}
\CommentTok{# standardize the posterior, so it sums to 1}
\NormalTok{posterior <-}\StringTok{ }\NormalTok{unstd.posterior }\OperatorTok{/}\StringTok{ }\KeywordTok{sum}\NormalTok{(unstd.posterior)}

\CommentTok{#plot}
\NormalTok{tylle_plot <-}\StringTok{ }\KeywordTok{plot}\NormalTok{( grid , posterior , }\DataTypeTok{type=}\StringTok{"b"}\NormalTok{ ,}
    \DataTypeTok{xlab=}\StringTok{"probability of correct"}\NormalTok{ , }\DataTypeTok{ylab=}\StringTok{"posterior probability"}\NormalTok{ )}
\KeywordTok{mtext}\NormalTok{(}\StringTok{"Tylle is stupid, sceptic prior"}\NormalTok{)}
\end{Highlighting}
\end{Shaded}

\includegraphics{Thea_Assignment2_Part1_files/figure-latex/unnamed-chunk-14-2.pdf}


\end{document}
